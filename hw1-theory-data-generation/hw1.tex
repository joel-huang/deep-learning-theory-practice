\documentclass[9pt,twocolumn]{article}

\usepackage[margin=0.8in,bottom=1.25in,columnsep=.4in]{geometry}
\usepackage{amsmath}
\usepackage{amssymb}
\usepackage{listings}
\usepackage{color}
\usepackage{cite}

\title{
	50.039 Theory and Practice of Deep Learning\\
	Theory Homework 1
}

\author{Joel Huang 1002530}
\date{\today}

\begin{document}
\maketitle

\section*{Drawing data from a distribution of $(x,y)$}
\subsection*{Problem definition}
\begin{itemize}
\item We need to draw pairs of data $x$ and binary labels $y=0$ or $y=1$.
\item There are two Gaussians with cluster index $c(x)=1$ and $c(x)=2$, each with a different probability of getting class $y=0$ and $y=1$.
\item $x$ follows the Gaussian distributions of the two clusters.
\end{itemize}

\subsection*{Formulation}
\begin{itemize}
\item $x$ and $y$ are independent: $P(x,y)=P(x)\,P(y)$.
\item $y$ is only dependent on cluster index $c(x)$. So $P(y\,|\,x,c(x))=P(y\,|\,c(x))$.
\item $P(x)$ is the product of the distribution of the data points given the cluster, and the probability of choosing the cluster:
\begin{equation*}
\begin{split}
	P(x) = & f(x\,|\,c(x))\,P(c(x))\\
	= & f(x\,|\,c(x)=1)\,P(c(x)=1)\\
	& + f(x\,|\,c(x)=2)\,P(c(x)=2)
\end{split}
\end{equation*}
\item $P(y=0)$ is the product of the probability of generating label $y=0$ given the cluster, and the probability of choosing the cluster:
\begin{equation*}
\begin{split}
	P(y=0) =&P(y=0\,|\,c(x)=1)\,P(c(x)=1) \\
	       &+P(y=0\,|\,c(x)=2)\,P(c(x)=2)
\end{split}
\end{equation*}
\item $P(y=1)$ can be expressed in terms of $P(y=0)$ because the labels are binary. Using the relationship $P(y=1\,|\,c(x)) = 1-P(y=0\,|\,c(x))$:
\begin{equation*}
\begin{split}
	P(y=1) = &(1-P(y=0\,|\,c(x)=1))\,P(c(x)=1) \\
	       & + (1-P(y=0\,|\,c(x)=2))\,P(c(x)=2)
\end{split}
\end{equation*}
\end{itemize}

\subsection*{Expressing $P(x,y)$}
\begin{equation*}
\begin{split}
	P(x,y) & = P(x)P(y) \\
	& =
\begin{cases}
	P(x)P(y=0),& \text{for} \,y=0\\
	P(x)P(y=1),& \text{for} \,y=1\\
	0         ,& \text{otherwise}
\end{cases}
\end{split}
\end{equation*}
For $y=0$,
\begin{equation*}
\begin{split}
	P(x,y=0) = & P(x)P(y=0) \\
	= & f(x\,|\,c(x))\,P(c(x))\\
	& \,\cdot [(P(y=0\,|\,x,c(x)=1)\,P(c(x)=1)\\
	& + P(y=0\,|\,x,c(x)=2)\,P(c(x)=2)]\\
	= & [f(x\,|\,c(x)=1)\,P(c(x)=1)\\
	& + f(x\,|\,c(x)=2)\,P(c(x)=2)]\\
	& \,\cdot [0.2 \cdot P(c(x)=1) + 0.7 \cdot P(c(x)=2)]
\end{split}
\end{equation*}
Similarly for $y=1$,
\begin{equation*}
\begin{split}
	P(x,y=1) = & P(x)P(y=1) \\
	= & f(x\,|\,c(x))\,P(c(x))\\
	& \,\cdot [(1-(P(y=0\,|\,x,c(x)=1))\,P(c(x)=1)\\
	& + (1-P(y=0\,|\,x,c(x)=2))\,P(c(x)=2)]\\
	= & [f(x\,|\,c(x)=1)\,P(c(x)=1)\\
	& + f(x\,|\,c(x)=2)\,P(c(x)=2)]\\
	& \,\cdot [0.8 \cdot P(c(x)=1) + 0.3 \cdot P(c(x)=2)]
\end{split}
\end{equation*}
\end{document}